\documentclass[pdf, unicode, 12pt, a4paper,oneside,fleqn]{article}
\usepackage{styles/log-style}
\begin{document}

\begin{titlepage}
    \begin{center}
    \bfseries
    
    {\Large Московский авиационный институт\\ (национальный исследовательский университет)
    
    }
  
    \vspace{48pt}
    
    {\large Факультет информационных технологий и прикладной математики
    }
    
    \vspace{36pt}
    
    
    {\large Кафедра вычислительной математики и~программирования
    
    }
    
    
    \vspace{48pt}
    
    Лабораторная работа №1 по курсу \enquote{Дискретный анализ}
    
    \end{center}
    
    \vspace{72pt}
    
    \begin{flushright}
    \begin{tabular}{rl}
    Студент: & Б.\,И. Вепринцев \\
    Преподаватель: & А.\,А. Кухтичев \\
    Группа: & М8О-207Б \\
    Дата: & \\
    Оценка: & \\
    Подпись: & \\
    \end{tabular}
    \end{flushright}
    
    \vfill
    
    \begin{center}
    \bfseries
    Москва, \the\year
    \end{center}
    \end{titlepage}
    
    \pagebreak

    \CWHeader{Лабораторная работа \textnumero 1}

    \CWProblem{
    Требуется разработать программу, осуществляющую ввод пар \enquote{ключ-значение}, их 
    упорядочивание по возрастанию ключа указанным алгоритмом сортировки за линейное время и вывод отсортированной последовательности.

    {\bfseries Вариант сортировки:} Поразрядная сортировка.

    {\bfseries Вариант ключа:} { \normalfont\ttfamily Автомобильные номера в формате А 999 BC (используются буквы латинского алфавита). }

    {\bfseries Вариант значения:} { \normalfont\ttfamily Числа от $0$ до $2^{64} - 1$. }
    }
    \pagebreak
    \section{Описание}
    Поразрядная сортировка является повторением сортировки подсчетом для каждого разряда или сразу нескольких разрядов.
Как сказано в \cite{Kormen}: \enquote{основная идея сортировки подсчетом заключается в том, чтобы для каждого входного 
элемента $x$ определить количество элементов, которые меньше $x$}.\\
Число элементов меньше $x$ будет обозначать место элемента в выходном отсортированном массиве.\\

К особенностям алгоритма относится то, что необходимо определить верхнюю границу $k$ допустимых значений ключей элементов, входящих во входной массив. Если $k = O(n) \Rightarrow k = \Theta(n)$. При этом является стабильной сортировкой и ни одна пара элементов при этом не сравнивается друг с другом.\\
К минусам можно отнести то, что требует памяти для 2 дополнительных массивов(помимо исходного массива) размерности $k$ и $n$ соответственно.\\

{\it Алгоритм:}\\


{   \textbf{for} $a \leftarrow 0 \: \textbf{to} \: max digit$\\
    \it Counting Sort(A)}:\\
\textbf{for} $i \leftarrow 0 \: \textbf{to} \: k$\\
  {\hspace*{0.5cm}$ C[i] \leftarrow 0$}\\
\textbf{for} $j \leftarrow 1 \: \textbf{to} \: length[A]$\\
  {\hspace*{0.5cm}$ C[A[j]] \leftarrow C[A[j]] + 1$}\\
\textbf{for} $i \leftarrow 1 \: \textbf{to} \: k$\\
  {\hspace*{0.5cm} $ C[i] \leftarrow C[i] + C[i-1]$}\\
\textbf{for} $j \leftarrow length[A] \: \textbf{downto} \: 1$\\
  {\hspace*{0.5cm}$ B[C[A[j]]] \leftarrow A[j]$}\\
  {\hspace*{0.5cm}$ C[A[j]] \leftarrow C[A[j]]-1$}\\

\pagebreak

\section{Исходный код}

На каждой непустой строке входного файла располагается пара \enquote{ключ-значение}, поэтому создадим новую 
структуру $pare$, в которой будем хранить три цифры номера, три буквы по отдельности, а также значение. 

\begin{lstlisting}[language=C]
  #include <stdio.h>
  #include <stdlib.h>
  #include <math.h>
  

typedef struct 
{
    int first;
    int second;
    int third;
    int number;
    unsigned long long int value;
}pare;

int max_number(int *data, int number)
void counting_sort(pare* data_full, int number, int number_elem)
void print(pare* result, int number_of_elem)

\end{lstlisting}

\pagebreak

{\it Описание методов и функций:}\\
\begin{longtable}{|p{7.5cm}|p{7.5cm}|}
\hline
\rowcolor{lightgray}
\multicolumn{2}{|c|} {main.c}\\
\hline
$int\  max\_number(int* \ data, int \ number)$&Нахождение максимальноо числа\\
\hline
$void\  counting\_sort(pare* \ data\_full, int \ number, int \ number\_elem )$&Метод сортировки подсчетом для каждого вида\\
\hline
$void\  print(pare*\ result, int \ number\_of\_elem)$&Метод печати ответа\\
\hline
$int\ main()$&Основная функция работы\\
\hline
\rowcolor{lightgray}

\end{longtable}

Помимо этого у нас имеется файл {\it gen\_test.py}, в котором реализован генератор тестов для программы, принимающий на вход имя тестового файла и количество тестовых строк.

\pagebreak

\section{Консоль}
\begin{alltt}

  boris@boris-VirtualBox:~/study/lab_da/lab_da1$ gcc -std=c99 -g -Wall -o lab lab1_2.c
  boris@boris-VirtualBox:~/study/lab_da/lab_da1$ python3 gen_test.py
  boris@boris-VirtualBox:~/study/lab_da/lab_da1$ cat test.txt
  V 637 ZQ 1935
  S 545 MI 257315
  G 956 TN 199115
  J 843 CQ 72494
  L 725 KB 28
  V 752 IH 4442
  O 464 HS 55937
  P 598 JT 99519
  Z 295 HJ 523399
  X 465 AH 85
  
  boris@boris-VirtualBox:~/study/lab_da/lab_da1$ ./lab < test.txt
  G 956 TN        199115
  J 843 CQ        72494
  L 725 KB        28
  O 464 HS        55937
  P 598 JT        99519
  S 545 MI        257315
  V 637 ZQ        1935
  V 752 IH        4442
  X 465 AH        85
  Z 295 HJ        523399

\end{alltt}
\pagebreak
\section{Вывод}
Выполнив первую лабораторную работу по курсу \enquote{Дискретный анализ}, я получил полезные знания о том, как реализуется одна из 
сортировок за линейное время - поразраядная сортировка, и приобрел свой первый опыт разработки в условиях ограниченной памяти и времени выполения. \\
Так же в результате данной работы я понял, что лучше использовать уже готовые решения, нежели заново изобретать велосипед. В частности это
касается самых базовых вещей, например таких как вектор. Еще в течении работы над этой лабораторной, я укрепил свои навыки работы с отладчиком gdb.\\


\end{document}