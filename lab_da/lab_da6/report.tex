\documentclass[pdf, unicode, 12pt, a4paper,oneside,fleqn]{article}
\usepackage{styles/log-style}
\begin{document}

\begin{titlepage}
    \begin{center}
    \bfseries
    
    {\Large Московский авиационный институт\\ (национальный исследовательский университет)
    
    }
    
    \vspace{48pt}
    
    {\large Факультет информационных технологий и прикладной математики
    }
    
    \vspace{36pt}
    
    
    {\large Кафедра вычислительной математики и~программирования
    
    }
    
    
    \vspace{48pt}
    
    Лабораторная работа №1 по курсу \enquote{Дискретный анализ}
    
    \end{center}
    
    \vspace{72pt}
    
    \begin{flushright}
    \begin{tabular}{rl}
    Студент: & М.\,А. Бронников \\
    Преподаватель: & А.\,А. Кухтичев \\
    Группа: & М8О-207Б \\
    Дата: & \\
    Оценка: & \\
    Подпись: & \\
    \end{tabular}
    \end{flushright}
    
    \vfill
    
    \begin{center}
    \bfseries
    Москва, \the\year
    \end{center}
    \end{titlepage}
    
    \pagebreak

    \CWHeader{Лабораторная работа \textnumero 1}

\CWProblem{
Требуется разработать программу, осуществляющую ввод пар \enquote{ключ-значение}, их 
упорядочивание по возрастанию ключа указанным алгоритмом сортировки за линейное время и вывод отсортированной последовательности.

{\bfseries Вариант сортировки:} Поразрядная сортировка.

{\bfseries Вариант ключа:} { \normalfont\ttfamily Автомобильные номера в формате А 999 BC (используются буквы латинского алфавита). }

{\bfseries Вариант значения:} { \normalfont\ttfamily Числа от $0$ до $2^{64} - 1$. }
}
\pagebreak
\end{document}